\chapter{Cache Models}\label{section:3.8}
Some values in a database are know to change slower than others. To improve
performance, many developers like to cache often-used data to avoid making
unnecessary trips back to the database. SQLMap provides its own caching
system, that you configure through a \tt{<cacheModel>} element.

The results from a query Mapped Statement can be cached simply by specifying
the \tt{cacheModel} parameter in the statement tag (seen above). A cache model
is a configured cache that is defined within your DataMapper configuration
file. Cache models are configured using the \tt{cacheModel} element as
follows:

\begin{example}\label{example:3.45}
Configuring a cache using the Cache Model element
\begin{verbatim}
<cacheModel id="product-cache" implementation="LRU" >
  <flushInterval hours="24"/>
  <flushOnExecute  statement="insertProduct"/>
  <flushOnExecute  statement="updateProduct"/>
  <flushOnExecute  statement="deleteProduct"/>
  <property name="CacheSize" value="100"/>
</cacheModel>
\end{verbatim}
\end{example}

The cache model above will create an instance of a cache named
``product-cache'' that uses a Least Recently Used (LRU) implementation. The
value of the \tt{type} attribute is either a class name, or an alias for one
of the included implementations (see below). Based on the flush elements
specified within the cache model, this cache will be flushed every 24 hours.
There can be only one flush interval element and it can be set using hours,
minutes, seconds or milliseconds. In addition the cache will be flushed
whenever the \tt{insertProduct}, \tt{updateProduct}, or \tt{deleteProduct}
mapped statements are executed. There can be any number of ``flush on
execute'' elements specified for a cache. Some cache implementations may need
additional properties, such as the ``cache-size'' property demonstrated above.
In the case of the LRU cache, the size determines the number of entries to
store in the cache. Once a cache model is configured, you can specify the
cache model to be used by a mapped statement, for example:

\begin{example}\label{example:3.46}
Specifying a Cache Model from a Mapped Statement
\begin{verbatim}
<statement id="getProductList" cacheModel="product-cache">
  select * from PRODUCT where PRD_CAT_ID = #value#
</statement>
\end{verbatim}
\end{example}

\section{Cache Implementation}
The cache model uses a pluggable framework for supporting different types of
caches. The choice of cache is specified in the ``implementation'' attribute
of the \tt{cacheModel} element as discussed above. The class name specified
must be an implementation of the \tt{ISqlMapCache} interface, or one of the
two aliases discussed below. Further configuration parameters can be passed to
the implementation via the property elements contained within the body of the
\tt{cacheModel}. Currently there are 2 implementations included with the PHP
distribution.

\subsection{Least Recently Used [LRU] Cache} The LRU cache implementation uses
an Least Recently Used algorithm to determines how objects are automatically
removed from the cache. When the cache becomes over full, the object that was
accessed least recently will be removed from the cache. This way, if there is
a particular object that is often referred to, it will stay in the cache with
the least chance of being removed. The LRU cache makes a good choice for
applications that have patterns of usage where certain objects may be popular
to one or more users over a longer period of time (e.g. navigating back and
forth between paginated lists, popular search keys etc.).

The LRU implementation is configured as follows:
\begin{example}\label{example:3.48}
Configuring a LRU type cache
\begin{verbatim}
<cacheModel id="product-cache"  implementation="LRU" >
  <flushInterval hours="24"/>
  <flushOnExecute  statement="insertProduct"/>
  <flushOnExecute  statement="updateProduct"/>
  <flushOnExecute  statement="deleteProduct"/>
   <property name="CacheSize" value="100"/>
</cacheModel>
\end{verbatim}
\end{example}

Only a single property is recognized by the LRU cache implementation. This
property, named \tt{CacheSize} must be set to an integer value representing
the maximum number of objects to hold in the cache at once. An important thing
to remember here is that an object can be anything from a single string
instance to an array of object. So take care not to store too much in your
cache and risk running out of memory and disk space.

\subsection{FIFO Cache}
The FIFO cache implementation uses an First In First Out algorithm to
determines how objects are automatically removed from the cache. When the
cache becomes over full, the oldest object will be removed from the cache. The
FIFO cache is good for usage patterns where a particular query will be
referenced a few times in quick succession, but then possibly not for some
time later.

The FIFO implementation is configured as follows:

\begin{example}\label{example:3.49}
Configuring a FIFO type cache
\begin{verbatim}
<cacheModel id="product-cache" implementation="FIFO" >
  <flushInterval hours="24"/>
  <flushOnExecute  statement="insertProduct"/>
  <flushOnExecute  statement="updateProduct"/>
  <flushOnExecute  statement="deleteProduct"/>
  <property name="CacheSize" value="100"/>
</cacheModel>
\end{verbatim}
\end{example}

Only a single property is recognized by the FIFO cache implementation. This
property, named \tt{CacheSize} must be set to an integer value representing
the maximum number of objects to hold in the cache at once. An important thing
to remember here is that an object can be anything from a single String
instance to an array of object. So take care not to store too much in your
cache and risk running out of memory and disk space.
