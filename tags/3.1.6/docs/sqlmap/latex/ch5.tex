\chapter{Result Maps}\label{section:3.5}
Chapter~\ref{section:3.4} describes Parameter Maps and Inline parameters,
which map object properties to parameters in a database query. Result Maps
finish the job by mapping the result of a database query (a set of columns) to
object properties. Next to Mapped Statements, the Result Map is probably one
of the most commonly used and most important features to understand.

A Result Map lets you control how data is extracted from the result of a
query, and how the columns are mapped to object properties. A Result Map can
describe the column type, a null value replacement, and complex property
mappings including Collections. Example~\ref{example:3.24} shows the structure
of a \tt{<resultMap>} element.

\begin{example}\label{example:3.24}
The structure of a \tt{<resultMap>} element.
\begin{verbatim}
<resultMap id="resultMapIdentifier"
           [class="class.name"]
           [extends="[sqlMapNamespace.]resultMapId"]>

   <result property="propertyName"
           column="columnName"
           [columnIndex="columnIndex"]
           [dbType="databaseType"]
           [type="propertyCLRType"]
           [resultMapping="resultMapName"]
           [nullValue="nullValueReplacement"]
           [select="someOtherStatementName"]
           [lazyLoad="true|false"]
           [typeHandler="class.name"]
   />
   <result ... .../>
   <result ... .../>
</resultMap>
\end{verbatim}
\end{example}

In Example~\ref{example:3.24}, the [brackets] indicate optional attributes.
The \tt{id} attribute is required and provides a name for the statement to
reference. The \tt{class} attribute is also required, and specifies the full
name of a PHP class. This is the class that will be instantiated and populated
based on the result mappings it contains.

The \tt{resultMap} can contain any number of property mappings that map object
properties to the columns of a result element. The property mappings are
applied, and the columns are read, in the order that they are defined.
Maintaining the element order ensures consistent results between different
drivers and providers.

\begin{mybox}{Note:}
As with parameter classes, the result class must be a PHP class object or
array instance.
\end{mybox}

\section{Extending \tt{resultMaps}}
The optional \tt{extends} attribute can be set to the name of another
\tt{resultMap} upon which to base this \tt{resultMap}. All properties of the
``super'' \tt{resultMap} will be included as part of this \tt{resultMap}, and
values from the ``super'' \tt{resultMap} are set before any values specified
by this \tt{resultMap}. The effect is similar to extending a class.

\begin{mybox}{Tip:}
The ``super'' \tt{resultMap} must be defined in the file before the extending
\tt{resultMap}. The classes for the super and sub \tt{resultMaps} need not be
the same, and do not need to be related in any way.
\end{mybox}

\section{\tt{<resultMap>} attributes}
The \tt{<resultMap>} element accepts three attributes: \tt{id} (required),
\tt{class} (optional), and \tt{extends} (optional).

\subsection{\tt{id} attribute}
The required \tt{id} attribute provides a unique identifier for the
\tt{<resultMap>} within this Data Map.

\subsection{\tt{class} attribute}
The optional \tt{class} attribute specifies an object class to use with this
\tt{<resultMap>}. The full classname must be specified. Any class can be used.

\begin{mybox}{Note:}
As with parameter classes, the result class must be a PHP class object or
array instance.
\end{mybox}

\subsection{\tt{extends} attribute}
The optional \tt{extends} attribute allows the result map to inherit all of
the properties of the ``super'' \tt{resultMap} that it extends.

\section{\tt{<result>} Elements}

The \tt{<resultMap>} element holds one or more \tt{<result>} child elements
that map SQL result sets to object properties.

\subsection{\tt{property} attribute}
The \tt{property} attribute is the name of a property of the result object
that will be returned by the Mapped Statement. The name can be used more than
once depending on the number of times it is needed to populate the results.

\subsection{\tt{column} attribute}
The \tt{column} attribute value is the name of the column in the result set
from which the value will be used to populate the property.

\subsection{\tt{columnIndex} attribute}
The \tt{columnIndex} attribute value is the index of the column in the
ResultSet from which the value will be used to populate the object property.
This is not likely needed in 99\% of applications and sacrifices
maintainability and readability for speed. Some providers may not realize any
performance benefit, while others will speed up dramatically.

\subsection{\tt{dbType} attribute}
The \tt{dbType} attribute is used to explicitly specify the database column
type of the ResultSet column that will be used to populate the object
property. Although Result Maps do not have the same difficulties with null
values, specifying the type can be useful for certain mapping types such as
Date properties. Because an application language has one Date value type and
SQL databases may have many (usually at least 3), specifying the date may
become necessary in some cases to ensure that dates (or other types) are set
correctly. Similarly, String types may be populated by a \tt{VarChar},
\tt{Char} or \tt{CLOB}, so specifying the type might be needed in those cases
too.

\subsection{\tt{type} attribute}
The \tt{type} attribute is used to explicitly specify the property type of the
parameter to be set. If the attribute \tt{type} is not set and the framework
cannot otherwise determine the type, the type is assumed to be \tt{StdObject}.

\subsection{\tt{resultMapping} attribute}
The \tt{resultMapping} attribute can be set to the name of another
\tt{resultMap} used to fill the property. If the \tt{resultMap} is in an other
mapping file, you must specified the fully qualified name as :

\begin{verbatim}
resultMapping="[namespace.sqlMap.]resultMappingId"

resultMapping="Newspaper"
<!--resultMapping with a fully qualified name.-->
resultMapping="LineItem.LineItem"
\end{verbatim}

\subsection{\tt{nullValue} attribute}
The \tt{nullValue} attribute can be set to any valid value (based on property
type). The \tt{nullValue} attribute is used to specify an outgoing null value
replacement. What this means is that when the value is detected in the object
property, a NULL will be written to the database (the opposite behavior of an
inbound null value replacement). This allows you to use a ``magic'' null
number in your application for types that do not support null values (such as
int, double, float). When these types of properties contain a matching null
value (say, $-9999$), a NULL will be written to the database instead of the
value.

If your database has a NULLABLE column, but you want your application to
represent NULL with a constant value, you can specify it in the Result Map as
shown in Example~\ref{example:3.25}.

\begin{example}\label{example:3.25}
Specifying a \tt{nullvalue} attribute in a Result Map
\begin{verbatim}
<resultMap id="get-product-result" class="product">
  <result property="id" column="PRD_ID"/>
  <result property="description" column="PRD_DESCRIPTION"/>
  <result property="subCode" column="PRD_SUB_CODE" nullValue="-9999"/>
</resultMap>
\end{verbatim}
\end{example}

In Example~\ref{example:3.25}, if PRD\_SUB\_CODE is read as NULL, then the
\tt{subCode} property will be set to the value of $-9999$. This allows you to
use a primitive type to represent a NULLABLE column in the database. Remember
that if you want this to work for queries as well as updates/inserts, you must
also specify the \tt{nullValue} in the Parameter Map (see,
Section~\ref{section:nullValueParameter}).

\subsection{\tt{select} attribute}
The \tt{select} attribute is used to describe a relationship between objects
and to automatically load complex (i.e. user defined) property types. The
value of the statement property must be the name of another mapped statement.
The value of the database column (the column attribute) that is defined in the
same property element as this statement attribute will be passed to the
related mapped statement as the parameter. More information about supported
primitive types and complex property mappings/relationships is discussed later
in this document. The \tt{lazyLoad} attribute can be specified with the
\tt{select}.

\subsection{\tt{lazyLoad} attribute}
Use the \tt{lazyLoad} attribute with the \tt{select} attribute to indicate
whether or not the select statement's results should be lazy loaded. This can
provide a performance boost by delaying the loading of the select statement's
results until they are needed/accessed.

\subsection{\tt{typeHandler} attribute}
The \tt{typeHandler} attribute allows the use of a Custom Type Handler (see
the Custom Type Handler in the following section). This allows you to extend
the DataMapper's capabilities in handling types that are specific to your
database provider, are not handled by your database provider, or just happen
to be a part of your application design. You can create custom type handlers
to deal with storing and retrieving booleans from your database for example.

\section{Custom Type Handlers}
A custom type handler allows you to extend the DataMapper's capabilities in
handling types that are specific to your database provider, not handled by
your database provider, or just happen to be part of your application design.
The SQLMap for PHP DataMapper provides an interface,
\tt{ITypeHandlerCallback}, for you to use in implementing your custom type
handler.

\begin{example}\label{example:3.26}
\tt{ITypeHandlerCallback} interface
\begin{verbatim}
interface ITypeHandlerCallback
{
    public function getParameter($object);

    public function getResult($string);

    public function createNewInstance();
}
\end{verbatim}
\end{example}

The \tt{getParameter} method allows you to process a \tt{<statement>}
parameter's value before it is added as an parameter. This enables you to do
any necessary type conversion and clean-up before the DataMapper gets to work.

The \tt{getResult} method allows you to process a database result value right
after it has been retrieved by the DataMapper and before it is used in your
\tt{resultClass}, \tt{resultMap}, or \tt{listClass}.

The \tt{createNewInstance} method allows the DataMapper to create new instance
of a particular type handled by this callback.

One scenario where custom type handlers are useful are the when you want to
use date time values in the database. First, consider a very basic TDateTime
class.

\begin{verbatim}
class TDateTime
{
    private $_datetime;

    public function __construct($datetime=null)
    {
        if(!is_null($datetime))
            $this->setDatetime($datetime);
    }

    public function getTimestamp()
    {
        return strtotime($this->getDatetime());
    }

    public function getDateTime()
    {
        return $this->_datetime;
    }

    public function setDateTime($value)
    {
        $this->_datetime = $value;
    }
}
\end{verbatim}

We can use a custom type handler to intercept result and parameter mapping
that uses the say ``data'' as one of its property type. The handler can be
written as follows.

\begin{example}\label{example:3.27}
A \tt{TDateTime} Type Handler
\begin{verbatim}
class TDateTimeHandler implements ITypeHandlerCallback
{
    public function getResult($string)
    {
        return new TDateTime($string);
    }

    public function getParameter($parameter)
    {
        if($parameter instanceof TDateTime)
            return $parameter->getTimestamp();
        else
            return $parameter;
    }

    public function createNewInstance()
    {
        return new TDateTime;
    }
}
\end{verbatim}
\end{example}

With our custom type handler we can use the handler in our SqlMaps. To do
that, we specify it as a basic \tt{<typeHandler>} for all \tt{date} types
mapped in our SqlMap files

\begin{example}\label{example:3.29}
\tt{<typeHandler>} in SqlMap.config
\begin{verbatim}
[Our SqlMap.config]

<typeHandlers>
 <typeHandler type="date" callback="TDateTimeHandler"/>
</typeHandlers>


[One of our SqlMap.xml files]
 <parameterMap id="boc-params">
  <parameter property="releasedDate" type="date"/>
 </parameterMap>

 <resultMap id="boc-result"  class="BudgetObjectCode">
  <result property="releasedDate" column="BOC_DATE" type="date"/>
 </resultMap>
\end{verbatim}
\end{example}

%3.5.5. Inheritance Mapping The iBATIS DataMapper supports the implementation
%of object-oriented inheritance (subclassing) in your object model. There are
%several developer options for mapping entity classes and subclasses to
%database results:
%
%resultMap for each class resultMap with submaps for a class hierarchy
%resultMap with extended resultMaps for each subclass You can use the most
%efficient mapping strategies from a SQL and query performance perspective when
%using the inheritance mappings of the DataMapper. To implement an inheritance
%mapping, the resultMap must define one or more columns in your query's
%resultset that will serve to identify which resultMap should be used to map
%each result record to a specific subclass. In many cases, you will use one
%column value for the DataMapper to use in identifying the proper resultMap and
%subclass. This column is known as a discriminator.
%
%For example, we have a table defined in a database that contains Document
%records. There are five table columns used to store Document IDs, Titles,
%Types, PageNumbers, and Cities. Perhaps this table belongs to a legacy
%database, and we need to create an application using this table with a domain
%model that defines a class hierarchy of different types of Documents. Or
%perhaps we are creating a new application and database and just want to
%persist the data found in a set of related classes into one table. In either
%case, the DataMapper's inheritance mapping feature can help.
%
%\begin{verbatim}
%// Database table Document
%CREATE TABLE [Documents] (
%    [Document_ID] [int] NOT NULL ,
%    [Document_Title] [varchar] (32) NULL ,
%    [Document_Type] [varchar] (32)  NULL ,
%    [Document_PageNumber] [int] NULL  ,
%    [Document_City] [varchar] (32)  NULL
%)
%\end{verbatim}
%
%To illustrate this, let's take a look at a few example classes shown below
%that have a relationship through inheritance and whose properties can be
%persisted into our Documents table. First, we have a base Document class that
%has Id and Title properties. Next, we have a Book class that inherits from
%Document and contains an additional property called PageNumber. Last, we have
%a Newspaper class that also inherits from Document and contains a City
%property.
%
%Example 3.30. Documents, Books, and Newspapers!
%
%\begin{verbatim}
%// C# class
%public class Document {
%  private int _id = -1;
%  private string _title = string.Empty;
%
%  public int Id
%  {
%    get { return _id; }
%    set { _id = value; }
%  }
%
%  public string Title
%  {
%    get { return _title; }
%    set { _title = value; }
%  }
%}
%
%public class Book : Document {
%  private int _pageNumber = -1;
%
%  public int PageNumber
%  {
%    get { return _pageNumber; }
%    set { _pageNumber = value; }
%  }
%}
%
%public class Newspaper : Document {
%  private string _city = string.Empty;
%
%  public string City
%  {
%    get { return _city; }
%    set { _city = value; }
%  }
%}
%\end{verbatim}
%
%Now that we have our classes and database table, we can start working on our
%mappings. We can create one <select> statement that returns all columns in the
%table. To help the DataMapper discriminate between the different Document
%records, we're going to indicate that the Document\_Type column holds values
%that will distinguish one record from another for mapping the results into our
%class hierarchy.
%
%\begin{verbatim}
%// Document mapping file
%<select id="GetAllDocument" resultMap="document">
%   select
%     Document_Id, Document_Title, Document_Type,
%     Document_PageNumber, Document_City
%   from Documents
%   order by Document_Type, Document_Id
%</select>
%
%<resultMap id="document" class="Document">
%  <result property="Id" column="Document_ID"/>
%  <result property="Title" column="Document_Title"/>
%  <discriminator column="Document_Type" type="string"/>
%  <subMap value="Book" resultMapping="book"/>
%  <subMap value="Newspaper" resultMapping="newspaper"/>
%</resultMap>
%
%<resultMap id="book" class="Book" extends="document">
%  <property="PageNumber" column="Document_PageNumber"/>
%</resultMap>
%
%<resultMap id="newspaper" class="Newspaper"  extends="document">
%  <property="City" column="Document_City"/>
%</resultMap>
%\end{verbatim}
%
%The DataMapper compares the data found in the discriminator column to the
%different <submap> values using the column value's string equivalence. Based
%on this string value, iBATIS DataMapper will use the resultMap named "Book" or
%"Newspaper" as defined in the <submap> elements or it will use the "super"
%resultMap "Document" if neither of the submap values satisfy the comparison.
%With these resultMaps, we can implement an object-oriented inheritance mapping
%to our database table.
%
%If you want to use custom logic, you can use the typeHandler attribute of the
%<discriminator> element to specify a custom type handler for the discriminator
%column.
%
%Example 3.31. Complex disciminator usage with Custom Type Handler
%
%\begin{verbatim}
%<alias>
%  <typeAlias alias="CustomInheritance"
%  type="IBatisNet.DataMapper.Test.Domain.CustomInheritance, IBatisNet.DataMapper.Test"/>
%</alias>
%
%<resultMaps>
%  <resultMap id="document-custom-formula" class="Document">
%    <result property="Id" column="Document_ID"/>
%    <result property="Title" column="Document_Title"/>
%    <discriminator column="Document_Type" typeHandler="CustomInheritance"/>
%    <subMap value="Book" resultMapping="book"/>
%    <subMap value="Newspaper" resultMapping="newspaper"/>
%  </resultMap>
%</resultMaps>
%\end{verbatim}
%
%The value of the typeHandler attribute specifies which of our classes
%implements the ITypeHandlerCallback interface. This interface furnishes a
%GetResult method for coding custom logic to read the column result value and
%return a value for the DataMapper to use in its comparison to the resultMap's
%defined <submap> values.
%
%Example 3.32. Example ITypeHandlerCallback interface implementation
%
%\begin{verbatim}
%public class CustomInheritance : ITypeHandlerCallback {
%  #region ITypeHandlerCallback members
%
%  public object ValueOf(string nullValue)
%  {
%    throw new NotImplementedException();
%  }
%
%  public object GetResult(IResultGetter getter)
%  {
%   string type = getter.Value.ToString();
%
%   if (type=="Monograph" || type=="Book")
%   {
%     return "Book";
%   }
%   else if (type=="Tabloid" || type=="Broadsheet" || type=="Newspaper")
%   {
%     return "Newspaper";
%   }
%   else
%   {
%     return "Document";
%   }
%
%  }
%
%  public void SetParameter(IParameterSetter setter, object parameter)
%  {
%   throw new NotImplementedException();
%  }
%  #endregion
%}
%\end{verbatim}

\section{Implicit Result Maps}
If the columns returned by a SQL statement match the result object, you may
not need an explicit Result Map. If you have control over the relational
schema, you might be able to name the columns so they also work as property
names. In Example~\ref{example:3.33}, the column names and property names
already match, so a result map is not needed.

\begin{example}\label{example:3.33}
A Mapped Statement that doesn't need a Result Map
\begin{verbatim}
<statement id="selectProduct" resultClass="Product">
  select
    id,
    description
  from PRODUCT
  where id = #value#
</statement>
\end{verbatim}
\end{example}

Another way to skip a result map is to use column aliasing to make the column
names match the properties names, as shown in Example~\ref{example:3.34}.

\begin{example}\label{example:3.34}
A Mapped Statement using column aliasing instead of a Result Map
\begin{verbatim}
<statement id="selectProduct" resultClass="Product">
  select
  PRD_ID as id,
  PRD_DESCRIPTION as description
  from PRODUCT
  where PRD_ID = #value#
</statement>
\end{verbatim}
\end{example}

Of course, these techniques will not work if you need to specify a column
type, a null value, or any other property attributes.

\section{Primitive Results (i.e. String, Integer, Boolean)}
Many times, we don't need to return an object with multiple properties. We
just need a string, integer, boolean, and so forth. If you don't need to
populate an object, SQLMap can return one of the primitive types instead. If
you just need the value, you can use a primitive type as a result class, as
shown in Example~\ref{example:3.35}.

\begin{example}\label{example:3.35}
Selecting a primitive type
\begin{verbatim}
<select id="selectProductCount" resultClass="integer">
  select count(1)
  from PRODUCT
</select>
\end{verbatim}
\end{example}

\begin{example}\label{example:3.36}
Loading a simple list of product descriptions
\begin{verbatim}
<resultMap id="select-product-result" resultClass="System.String">
  <result property="value" column="PRD_DESCRIPTION"/>
</resultMap>
\end{verbatim}
\end{example}

\section{Maps with ResultMaps}
Instead of a rich object, sometimes all you might need is a simple key/value
list of the data, where each property is an entry on the list. If so, Result
Maps can populate an array instance as easily as property objects. The syntax
for using an array is identical to the rich object syntax. As shown in Example
~\ref{example:3.37}, only the result object changes.

\begin{example}\label{example:3.37}
Result Maps can use arrays.
\begin{verbatim}
<resultMap id="select-product-result" class="array">
  <result property="id" column="PRD_ID"/>
  <result property="code" column="PRD_CODE"/>
  <result property="description" column="PRD_DESCRIPTION"/>
  <result property="suggestedPrice" column="PRD_SUGGESTED_PRICE"/>
</resultMap>
\end{verbatim}
\end{example}

In Example~\ref{example:3.37}, an array instance would be created for each row
in the result set and populated with the Product data. The property name
attributes, like \tt{id}, \tt{code}, and so forth, would be the key of the
entry, and the value of the mapped columns would be the value of the entry.

As shown in Example~\ref{example:3.38}, you can also use an implicit Result
Map with an array type.

\begin{example}\label{example:3.38}
Implicit Result Maps can use arrays too.
\begin{verbatim}
<statement id="selectProductCount" resultClass="array">
  select * from PRODUCT
</statement>
\end{verbatim}
\end{example}

What set of entries is returned by Example~\ref{example:3.38} depends on what
columns are in the result set. If the set of column changes (because columns
are added or removed), the new set of entries would automatically be returned.

\begin{mybox}{Note:}
Certain providers may return column names in upper case or lower case format.
When accessing values with such a provider, you will have to pass the key name
in the expected case.
\end{mybox}

\section{Complex Properties}
In a relational database, one table will often refer to another. Likewise,
some of your business objects may include another object or list of objects.
Types that nest other types are called ``complex types''. You may not want a
statement to return a simple type, but a fully-formed complex type.

In the database, a related column is usually represented via a 1:1
relationship, or a 1:M relationship where the class that holds the complex
property is from the ``many side'' of the relationship and the property itself
is from the ``one side'' of the relationship. The column returned from the
database will not be the property we want; it is a key to be used in another
query.

From the framework's perspective, the problem is not so much loading a complex
type, but loading each ``complex property''. To solve this problem, you can
specify in the Result Map a statement to run to load a given property. In
Example~\ref{example:3.39}, the ``category'' property of the
``select-product-result'' element is a complex property.

\begin{example}\label{example:3.39}
A Result Map with a Complex Property
\begin{verbatim}
  <resultMap id="select-product-result" class="product">
    <result property="id" column="PRD_ID"/>
    <result property="description" column="PRD_DESCRIPTION"/>
    <result property="category" column="PRD_CAT_ID" select="selectCategory"/>
  </resultMap>

  <resultMap id="select-category-result" class="category">
    <result property="id" column="CAT_ID"/>
    <result property="description" column="CAT_DESCRIPTION"/>
  </resultMap>

  <select id="selectProduct" parameterClass="int" resultMap="select-product-result">
   select * from PRODUCT where PRD_ID = #value#
  </select>

  <select id="selectCategory" parameterClass="int" resultMap="select-category-result">
   select * from CATEGORY where CAT_ID = #value#
  </select>
\end{verbatim}
\end{example}

In Example~\ref{example:3.39}, the framework will use the ``selectCategory''
statement to populate the ``category'' property. The value of each category is
passed to the ``selectCategory'' statement, and the object returned is set to
the category property. When the process completes, each Product instance will
have the the appropriate category object instance set.

\section{Avoiding N+1 Selects (1:1)}
A problem with Example~\ref{example:3.39} may be that whenever you load a
Product, two statements execute: one for the Product and one for the Category.
For a single Product, this issue may seem trivial. But if you load 10
products, then 11 statements execute. For 100 Products, instead of one
statement product statement executing, a total of 101 statements execute. The
number of statements executing for Example~\ref{example:3.40} will always be
N+1: 100+1=101.

\begin{example}\label{example:3.40}
N+1 Selects (1:1)
\begin{verbatim}
  <resultMap id="select-product-result" class="product">
    <result property="id" column="PRD_ID"/>
    <result property="description" column="PRD_DESCRIPTION"/>
    <result property="category" column="PRD_CAT_ID" select="selectCategory"/>
  </resultMap>

  <resultMap id="select-category-result" class="category">
    <result property="id" column="CAT_ID"/>
    <result property="description" column="CAT_DESCRIPTION"/>
  </resultMap>

  <!-- This statement executes 1 time -->
  <select id="selectProducts" parameterClass="int" resultMap="select-product-result">
   select * from PRODUCT
  </select>

  <!-- This statement executes N times (once for each product returned above) -->
  <select id="selectCategory" parameterClass="int" resultMap="select-category-result">
   select * from CATEGORY where CAT_ID = #value#
  </select>

\end{verbatim}
\end{example}

One way to mitigate the problem is to cache the ``selectCategory'' statement .
We might have a hundred products, but there might only be five categories.
Instead of running a SQL query or stored procedure, the framework will return
the category object from it cache. A 101 statements would still run, but they
would not be hitting the database. (See Chapter~\ref{section:3.8} for more
about caches.)

Another solution is to use a standard SQL join to return the columns you need
from the another table. A join can bring all the columns we need over from the
database in a single query. When you have a nested object, you can reference
nested properties using a dotted notation, like ``category.description''.

Example~\ref{example:3.41} solves the same problem as
Example~\ref{example:3.40}, but uses a join instead of nested properties.

\begin{example}\label{example:3.41}
Resolving complex properties with a join
\begin{verbatim}
  <resultMap id="select-product-result" class="product">
    <result property="id" column="PRD_ID"/>
    <result property="description" column="PRD_DESCRIPTION"/>
    <result property="category" resultMapping="Category.CategoryResult" />
  </resultMap>

  <statement id="selectProduct" parameterClass="int" resultMap="select-product-result">
    select *
    from PRODUCT, CATEGORY
    where PRD_CAT_ID=CAT_ID
    and PRD_ID = #value#
  </statement>
\end{verbatim}
\end{example}

\begin{mybox}{Lazy Loading vs. Joins (1:1):}
It's important to note that using a join is not always better. If you are in a
situation where it is rare to access the related object (e.g. the category
property of the Product class) then it might actually be faster to avoid the
join and the unnecessary loading of all category properties. This is
especially true for database designs that involve outer joins or nullable
and/or non-indexed columns. In these situations it might be better to use the
sub-select solution with lazy loading enabled. The general rule of thumb is:
use the join if you're more likely going to access the associated properties
than not. Otherwise, only use it if lazy loading is not an option.
\\
\\
If you're having trouble deciding which way to go, don't worry. No matter
which way you go, you can always change it without impacting your application
source code. Example~\ref{example:3.40} and \ref{example:3.41} result in
exactly the same object graph and are loaded using the exact same method call
from the application. The only consideration is that if you were to enable
caching, then the using the separate select (not the join) solution could
result in a cached instance being returned. But more often than not, that
won't cause a problem (your application shouldn't be dependent on instance
level equality i.e. ``==='').
\end{mybox}

\section{Complex Collection Properties}
It is also possible to load properties that represent lists of complex
objects. In the database the data would be represented by a M:M relationship,
or a 1:M relationship where the class containing the list is on the ``one
side'' of the relationship and the objects in the list are on the ``many
side''. To load a \tt{TList} of objects, there is no change to the statement
(see example above). The only difference required to cause the SQLMap
DataMapper framework to load the property as a \tt{TList} is that the property
on the business object must be of type \tt{TList}. For example, if a Category
has a \tt{TList} of Product instances, the mapping would look like this
(assuming Category has a property called "ProductList" of \tt{TList}.):

\begin{example}\label{example:3.42}
Mapping that creates a list of complex objects
\begin{verbatim}
<resultMaps>

  <resultMap id="select-category-result" class="Category">
    <result property="Id" column="CAT_ID"/>
    <result property="Description" column="CAT_DESCRIPTION"/>
    <result property="ProductList" column="CAT_ID" select="selectProductsByCatId"/>
  </resultMap>

  <resultMap id="select-product-result" class="Product">
    <result property="Id" column="PRD_ID"/>
    <result property="Description" column="PRD_DESCRIPTION"/>
  </resultMap>
<resultMaps>

<statements>

  <statement id="selectCategory" parameterClass="int"
            resultMap="select-category-result">
    select * from CATEGORY where CAT_ID = #value#
  </statement>

  <statement id="selectProductsByCatId" parameterClass="int"
            resultMap="select-product-result">
    select * from PRODUCT where PRD_CAT_ID = #value#
  </statement>
</statements>
\end{verbatim}
\end{example}

\section{Avoiding N+1 Select Lists (1:M and M:N)}
This is similar to the 1:1 situation above, but is of even greater concern due
to the potentially large amount of data involved. The problem with the
solution above is that whenever you load a Category, two SQL statements are
actually being run (one for the Category and one for the list of associated
Products). This problem seems trivial when loading a single Category, but if
you were to run a query that loaded ten (10) Categories, a separate query
would be run for each Category to load its associated list of Products. This
results in eleven (11) queries total: one for the list of Categories and one
for each Category returned to load each related list of Products (N+1 or in
this case 10+1=11). To make this situation worse, we're dealing with
potentially large lists of data.

\begin{example}\label{example:3.43}
N+1 Select Lists (1:M and M:N)
\begin{verbatim}
<resultMaps>

  <resultMap id="select-category-result" class="Category">
    <result property="Id" column="CAT_ID"/>
    <result property="Description" column="CAT_DESCRIPTION"/>
    <result property="ProductList" column="CAT_ID" select="selectProductsByCatId"/>
  </resultMap>

  <resultMap id="select-product-result" class="Product">
    <result property="Id" column="PRD_ID"/>
    <result property="Description" column="PRD_DESCRIPTION"/>
  </resultMap>
<resultMaps>

<statements>

  <!-- This statement executes 1 time -->
  <statement id="selectCategory" parameterClass="int"
            resultMap="select-category-result">
    select * from CATEGORY where CAT_ID = #value#
  </statement>

  <!-- This statement executes N times (once for each category returned above)
       and returns a list of Products (1:M) -->
  <statement id="selectProductsByCatId" parameterClass="int"
            resultMap="select-product-result">
    select * from PRODUCT where PRD_CAT_ID = #value#
  </statement>
</statements>
\end{verbatim}
\end{example}

\subsection{1:N \& M:N Solution?}
Currently the feature that resolves this issue not implemented, but the
development discussions are active, and we expect it to be included in a
future release.

\begin{mybox}{Lazy Loading vs. Joins (1:M and M:N):}
As with the 1:1 situation described previously, it's important to note that
using a join is not always better. This is even more true for collection
properties than it was for individual value properties due to the greater
amount of data. If you are in a situation where it is rare to access the
related object (e.g. the ProductList property of the Category class) then it
might actually be faster to avoid the join and the unnecessary loading of the
list of products. This is especially true for database designs that involve
outer joins or nullable and/or non-indexed columns. In these situations it
might be better to use the sub-select solution with the lazy loading. The
general rule of thumb is: use the join if you're more likely going to access
the associated properties than not. Otherwise, only use it if lazy loading is
not an option.
\\
\\
As mentioned earlier, if you're having trouble deciding which way to go, don't
worry. No matter which way you go, you can always change it without impacting
your PHP code. The two examples above would result in exactly the same object
graph and are loaded using the exact same method call. The only consideration
is that if you were to enable caching, then the using the separate select (not
the join) solution could result in a cached instance being returned. But more
often than not, that won't cause a problem (your application should not be
dependent on instance level equality i.e. ``==='').
\end{mybox}

\section{Composite Keys or Multiple Complex Parameters Properties}
You might have noticed that in the above examples there is only a single key
being used as specified in the \tt{resultMap} by the \tt{column} attribute.
This would suggest that only a single column can be associated to a related
mapped statement. However, there is an alternate syntax that allows multiple
columns to be passed to the related mapped statement. This comes in handy for
situations where a composite key relationship exists, or even if you simply
want to use a parameter of some name other than \tt{\#value\#}. The alternate
syntax for the column attribute is simply \{param1=column1, param2=column2,
$\cdots$, paramN=columnN\}. Consider the example below where the PAYMENT table
is keyed by both Customer ID and Order ID:

\begin{example}\label{example:3.44}
Mapping a composite key
\begin{verbatim}
<resultMaps>
  <resultMap id="select-order-result" class="order">
    <result property="id" column="ORD_ID"/>
    <result property="customerId" column="ORD_CST_ID"/>
    ...
    <result property="payments" column="{itemId=ORD_ID, custId=ORD_CST_ID}"
      select="selectOrderPayments"/>
  </resultMap>
<resultMaps>

<statements>

  <statement id="selectOrderPayments" resultMap="select-payment-result">
    select * from PAYMENT
    where PAY_ORD_ID = #itemId#
    and PAY_CST_ID = #custId#
  </statement>
</statements>
\end{verbatim}
\end{example}

Optionally you can just specify the column names as long as they're in the
same order as the parameters. For example:
\begin{verbatim}
{ORD_ID, ORD_CST_ID}
\end{verbatim}

\begin{mybox}{Important!}
Currently the SQLMap DataMapper framework does not automatically resolve
circular relationships. Be aware of this when implementing parent/child
relationships (trees). An easy work around is to simply define a second result
map for one of the cases that does not load the parent object (or vice versa),
or use a join as described in the ``N+1 avoidance'' solutions.
\end{mybox}

\begin{mybox}{Note:}
Result Map names are always local to the Data Map definition file that they
are defined in. You can refer to a Result Map in another Data Map definition
file by prefixing the name of the Result Map with the namespace of the SqlMap
set in the \tt{<sqlMap>} root element.
\end{mybox}
